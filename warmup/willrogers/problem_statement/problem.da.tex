\problemname{Will Rogers-fænomenet}
Den amerikanske komiker Will Rogers (1879-1935) menes at have sagt om \emph{okierne}, de fattige migranter fra Oklahoma i 1920erne, at »da okierne  flyttede til Kalifornien, forbedrede de gennemsnitsintelligensen i begge stater«.

Dette øjensynlige paradoks -- at flytningen af et element fra en mængde til en anden kan medføre, at gennemsnittet i begge mængderne bliver større -- kaldes derfor sommetider Will Rogers-fænomenet.
Du skal skrive et program, som indlæser to grupper A og B, hver bestående af mindst to og højst ti positive heltal.
Dit program skal afgøre, om det er muligt at flytte et enkelt tal mellem grupperne og dermed øge gennemsnittet i begge grupper.
I så fald, hvilket tal skal flyttes?

\section*{Indlæsning}

Første linje består af to tal: antallet af tal i A og antallet af tal i B (begge mellem $1$ og $10$).
Derefter følger en linje med tallene i A og en linje med tallene i B.

Alle tal er mellem $1$ og $20$.

\section*{Udskrift}
Hvis det er muligt at flytte et tal mellem grupperne og dermed at øge gennemsnittet af begge, skriv en linje med det tal, der skal flyttes, og hvilken gruppe, det skal flyttes til.
Hvis der findes flere muligheder, så er det nok at angive en af dem.

Hvis det ikke er muligt, skriv \texttt{NEJ}.

\section*{Forklaring af eksemplerne}

I det første eksempel er gennemsnittene $3$ og $4$ inden flytning.

Efter at flytte tallet $3$ fra B til A er gennemsnittene $2{,}25$ og $4{,}333\ldots$.

I det andet eksempel kan fænomenet ikke forekomme.
Hvis man fx flytter tallet $5$ fra A til B, så vokser rigtignok gennemsnittet i A, men gennemsnittet i B forbliver uforandret. 
