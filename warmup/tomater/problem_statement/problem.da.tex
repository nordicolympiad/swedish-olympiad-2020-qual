\problemname{Tomater}

Interessant nok modner tomater hurtigere, når man lægger nogle allerede modne tomater ind blandt dem.
I denne opgave skal du simulere denne proces og regne ud, hvor mange modne tomater der er efter et vist stykke tid.

Antag, at $n$ tomater ligger på række og er nummererede fra $1$ til $n$.
Tre af tomaterne, nummer $t_1$, $t_2$ og $t_3$, er allerede modne, når simuleringen begynder på dag $0$.
Hver dag modnes de tomater, som ligger ved siden af en moden tomat.
Efter dag $1$ er altså nabotomaterne til de første tre modne tomater blevet modne, efter dag $2$ er nabotomaterne til de på dag~$1$ modnede tomater blevet modne, og så videre.

Skriv et program, som givet antallet af tomater $n$, antallet af dage $d$ og numrene $t_1$, $t_2$, $t_3$, regner ud, hvor mange modne tomater der er efter $d$ dage.

\section*{Indlæsning}

På første linje af inddata står de to tal $n$ ($3 \le n \le 100$) og $d$ ($1 \le d \le 100$).

På anden linje står numrene $t_1$, $t_2$ og $t_3$, alle forskellige og i intervallet $1 \dots n$.

\section*{Udskrift}
Skriv et enkelt tal: antallet af modne tomater efter $d$ dage.
