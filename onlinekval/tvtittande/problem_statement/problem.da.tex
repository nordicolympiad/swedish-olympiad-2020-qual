\problemname{Tv-kiggeri}
Børges venner elsker tv-serier og plejer at diskutere dem på deres fødselsdagsfester.
Børge føler sig ofte udenfor, fordi han ikke har set de samme serier som dem.

Børge er inviteret til fødselsdagsfest på visse dage og vil gå til dem alle.
Han ved i forvejen, hvilke tv-serier som bliver diskuteret til hvilke fester, og vil se se pågældende serier færdigt for at kunne snakke med.

Børge vil ikke kigge på tv mere end ti timer per dag, og han har ikke tid til at kigge på tv på de dage, hvor han skal på fest.

Han kan når so helst tagee en pause fra en tv-serie og fortsætte med at kigge en anden dag, men når han er på et fest, hvor serien bliver diskuteret, skal han have set den færdig.
Kan Børges forsæt gennemføres?

\section*{Indlæsning}
På første linje står de to heltal $n$ og $k$ ($1 \leq n,k \leq 10^5$), som angiver antallet af fester og antallet af tv-serier.
Tv-serierne er nummererede fra $1$ til $k$.

På næste linje står $k$ heltal, hvoraf det $i$te tal angiver længden av tv-serie nummer~$i$ målt i timer.
Ingen serie er længere end $10^6$ timer.

De følgende $n$ linjer bekriver festerne i rækkefølge.
Linje~$i$ begynder med to heltal $1 \leq d_i \leq 10^5$ og $c_i$, som angiver hvilketn dag festen finder sted og antallet af tv-serier, som vil blive diskuteret.
Derefter følger $c_i$ forskellige heltal på samme linje, nemlig de på festen diskuterede tv-serier.
Summen af alla $c_i$ overstiger ikke $10^5$.

Børge er højst inviteret til en fest per dag. 
Nu er det morgen på dag~$0$,  og Børge skal altså ikke til nogen fest i dag.

\section*{Udskrift}
Udskriv \texttt{Ja}, hvis det er muligt at se alle tv-serier færdigt inden de fester, hvor de vil blive diskuteret.
Udskriv \texttt{Nej}, hvis det ikke er muligt.

\section*{Pointgivning}
Din løsning bliver kørt på en række testfaldsgrupper.
For at opnå point for en gruppe, skal din løsning klare samtlige testfald i gruppen.

\noindent
\begin{tabular}{| l | l | l |}
\hline
Gruppe & Pointværdi & Begrænsninger \\ \hline
$1$   & $20$       & $n \leq 50$, $k \leq 50$ \\ \hline
$1$   & $40$       & $n \leq 1000$ \\ \hline
$2$   & $40$       & Ingen yderligere begrænsninger. \\ \hline
\end{tabular}

\section*{Forklaring af eksemplet}
Næste fest er om to dage, og der vil den 20 timer lange tv-serie~2 blive diskuteret.
Det kan Børge præcis nå ved at kigge på 10 timer hver dag i to dage.
Derefter skal han bruge 5 timer hver til at se hhv.\ serie~3 og 4, hvilket han klarer på dag~3.
På dag~5 kan han nøjes med at kigge på serie~1, fordi han allerede har set serie~2.
