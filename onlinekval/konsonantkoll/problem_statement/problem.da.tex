\problemname{Konsonantkontrol}
Din svenske veninde Frida elsker konsonanter!
Desværre leder det hende ofte i uføre, når hun skriver.
Hun elsker nemlig konsonanter så meget, at hun i sin iver ofte trykker alt for mange gange på tasten, når hun skal skrive dobbeltkonsonanter.

For at hjælpe Frida, skal du skrive et program, der fjerner de overskydende konsonanter.
På svensk er der 20 konsonanter
\textbf{bcdfghjklmnpqrstvwxz}.
Praktisk nok indeholder korrekt svensk tekst aldrig tre eller flere ens konsonanter i stræk.

\section*{Indlæsning}
Indlæsningen består af en enkelt linje, som bare indeholder små bogstaver (a-z) og muligvis mellemrum. 
Linjen er højst 1000 tegn lang.

\section*{Udskrift}
Udskriv samme linje uden de overflødige konsonanter, dvs. således at der ikke forekommer flere end to ens konsonanter efter hinanden.

\section*{Pointgivning}

Din løsning bliver testet på to grupper af testfald.
For at få point for en gruppe skal du klare samtlige testfald i gruppen.

\noindent
\begin{tabular}{| l | l | l |}
\hline
Gruppe & Point & Begrænsninger \\ \hline
1     & 50          &  Indlæsningen har ingen mellemrum, og der forekommer højst 3 konsonanter i stræk.\\ \hline
2     & 50         &  Ingen begrænsninger. \\ \hline
\end{tabular}

